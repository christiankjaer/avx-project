\documentclass{article}
\usepackage[utf8]{inputenc}
\usepackage[T1]{fontenc}

\title{Extending MLKit with vector instructions}
\author{Christian Kjær Larsen}

\begin{document}

\maketitle

\section{SIMD}

\subsection{Modern hardware support}

\subsection{GCC}

Intrinsic functions.

Generic vector operations

\subsection{Higher level languages}

For C-sharp, System.Numeric.Vectors

For Java, jdk.incubator.vector


\section{Implementation}


\subsection{Generic interface}


A make function from a familar type, typically a tuple of elements.
A read function to read the familiar type again.


\subsubsection{Arithmetic}

Elementwise operations
Scalar operations

\subsubsection{Comparisons}

Mask type with all and any

\subsubsection{Masks}

Blend

\subsubsection{Handling arrays of odd length}






\subsubsection{Vectorizing a simple function}


\subsection{Implementation in the MLKit}

Boxed representation that is available to the programmer and an internal unboxed type that is available to the optimizer.

\subsection{Data types}

Reuse register allocation of vector registers since they completely overlap with xmm. Just rename to ymm when they are used in a vector instruction.

\subsubsection{Primops}

Both boxed and unboxed version of intrinsics. The unboxed ones are not available the the programmer, but are used internally in the optimizer.

\subsubsection{Unboxing}

Since Standard ML is polymorphic, there might be polymorphic functions that can receive arguments of any type. This is typically achieved by having a uniform representation of values. Typically a box (pointer to the underlying data). There is a substantial penalty for this, since in order to do operations on the underlying value, there is a level of indirection. For our new vector type we will have to use 16 bytes of memory in addition to the pointer to represent something that typically fits in a register.

Any operation that works on a boxed representation will typically have to unbox the data, do the operation and box the result again. This is very costly on modern hardware.

In some sense
\[
    box (unbox\ x) \equiv x
\]
and
\[
    unbox (box\ x) \equiv x
\]


\end{document}
